\documentclass{article}
\usepackage{amsmath,amssymb,amsthm}
\usepackage{graphicx}
\usepackage[pdfborder= 0 0 0,citecolor=black,linkcolor=black,colorlinks=true,bookmarksopen=true]{hyperref}
%\usepackage[notref,notcite]{showkeys}
\usepackage{caption}
\usepackage{subcaption}
\usepackage{algorithm}
\usepackage{algorithmicx}
\usepackage{algpseudocode}
\usepackage{tikz}
\usepackage{tikz-3dplot}
\usepackage{pgfplots}
\usepackage{authblk}
\usepackage[pagewise]{lineno}
\usepackage[square]{natbib}
\usetikzlibrary{plotmarks}
\author[1,*]{Weller, Browne, Budd, Cullen}
\title{Numerical solution of the optimal transport problem for mesh generation on the sphere}
\makeatletter \AtBeginDocument{ \hypersetup{pdftitle= {\@title},pdfauthor= {\@author}}} \makeatother

\date{\today}

\newtheorem{theorem}{Theorem}
\newtheorem{corollary}{Corollary}
\theoremstyle{definition}
\newtheorem{defn}{Definition}
\begin{document}
\linenumbers
\maketitle

The equidistribution principle does not lead to a well-defined problem for mesh generation on the sphere. Indeed this problem is ill-posed in more than 1 dimension and so requires the imposition of an extra constraint on the mesh we wish to find. For the resulting applications, such as interpolation or the numerical solution of a PDE, we postulate that a mesh which is as close as possible to the original computational mesh will be appropriate and effective. This is an \emph{optimal transport} constraint - we seek to minimise the distance between the original and resulting meshes in a certain measure which we shall discuss subsequently. We write this minimization problem

\[\min_{\mathbf{x}\in \Omega_p} d(\mathbf{\xi},\mathbf{x})^2\]
where $d$ is the distance metric between the two meshes. In Cartesian space $[0,1]^n$ this metric can simply be the sum of the Euclidean distance between all of the corresponding points defining the meshes. On the sphere $\mathbf{S}^2$, the appropriate metric is the Riemannian distance on the surface of the sphere between all of the corresponding points defining the meshes.

In Euclidean space the existence and uniqueness of the equidistribution principle subject to the optimal transport constraint follows as a direct result from \citet{Brenier1991}. On the sphere we appeal to the generalised version of Brenier's theorem given by \citet{McCann2001}, a detailed discussion of which is given in \citet{villani2003}.

\begin{defn}[$c$-convex function]
The $c$-transform $\phi^c$ of a function $\phi: \mathbf{S}^2 \to \mathbf{R}$ is defined as 
\[\phi^c (x) = \sup_{\xi\in\mathbf{S}^2} \{-c(\xi,x) - \phi(\xi)\}.\]
The function $\phi$ is said to be \emph{$c$-convex}, or cost-convex, if 
$(\phi^c)^c=\phi.$
\end{defn}

\begin{theorem}[\citet{McCann2001}]\label{mccannthm}
Let $M$ be a connected, complete smooth Riemannian manifold, equipped with its standard volume measure $dx$. Let $\mu, \nu$ be two probability measures on $M$ with compact support, and let the objective function $c(\xi,x)$ be equal to $d(\mathbf{\xi},\mathbf{x})^2$, where $d$ is the geodesic distance on $M$. Further, assume that $\mu$ is absolutely continuous with respect to the volume measure on $M$. Then, the Monge-Kantorovich mass transportation problem between $\mu$ and $\nu$ admits a unique optimal transport map $T$ with
\begin{equation}\label{t1}T(\mu) = \nu\end{equation}
such that
%\[\pi = (I \times T)\#\mu\]
%where
\begin{equation}\label{t2}\mathbf{x}=T(\mathbf{\xi}) = \exp_\mathbf{\xi} [\nabla \phi(\mathbf{\xi})]\end{equation}
for some $d^2/2$-convex potential $\phi$. 
\end{theorem}

\begin{corollary}
There exists a unique, optimally transported mesh on the sphere that satisfies the equidistribution principle. Moreover, that mesh is defined by a $c$-convex scalar potential function.
\end{corollary}

\begin{proof}
Clearly $M=S^2$ satisfies the conditions on $M$ in \autoref{mccannthm}. The first probability measure of interest, $\mu$, we define to be the lack of equidistribution under the original computational mesh, i.e.
\begin{equation}
\mu = \sum_{i=1}^{N}w_i \delta_i(\xi -\xi_i)
\end{equation}
where $N$ is the total number of cells in the mesh, $\delta_i(\xi -\xi_i)$ is the Dirac delta function located at the centre of cell $i$, $\xi_i$. $w_i$ is a weighting function which measures the local lack of equidistribution, i.e.
\begin{equation}
w_i = \int_{\xi_i \in \Omega_C} m(\mathbf{\xi}) \mathrm{d}\mathbf{\xi} - \frac{1}{N}\int_{\Omega_C} m(\mathbf{\xi}) \mathrm{d}\mathbf{\xi}
\end{equation}


%\begin{equation}
%\mu = m(\mathbf{\xi}) - \int_{\Omega_C}m(\xi) \ \mathrm{d}\mathbf{\xi}.
%\end{equation}
The target probability measure is one of equidistribution under the desired mesh, i.e.
\begin{equation}
\nu = \sum_{i=1}^{N}\overline{w_i} \delta_i(x -x_i)
\end{equation}
where $x_i$ is the centre of the new cell $i$, and the weighting
\begin{equation}
\overline{w_i} = \int_{x_i \in \Omega_p} |\nabla \mathbf{x}|m(\mathbf{x}) \mathrm{d}\mathbf{x}% - \frac{1}{N}\int_{\Omega_p} m(\mathbf{x}) \mathrm{d}\mathbf{x}
\end{equation}
satisfies 
\begin{equation}
\overline{w_i} =\frac{1}{N}\int_{\Omega_p} m(\mathbf{x}) \mathrm{d}\mathbf{x} \qquad \forall i \in \{1,\ldots,N\}.
\end{equation}

%\begin{equation}
%\nu = |\nabla \mathbf{x}|m(\mathbf{x}) - \int_{\Omega_p}m(\mathbf{x}) \ \mathrm{d}\mathbf{x}.
%\end{equation}
As $M=S^2$ these are trivially compactly supported. The monitor function $m(\mathbf{x})$ must be sufficiently smooth such that $\mu$ is absolutely continuous.
Hence by \autoref{mccannthm} we have that there exists a unique solution, $T$, to the mass transportation problem between $\mu$ and $\nu$. Under this solution, $\nu$ satisfies the equidistribution principle.

The map $T$ \textit{pushes-forward} the measure $\mu$ onto the desired solution $\nu$ via the exponential map. This exponential map is defined by the gradient of the $d^2/2$-convex potential $\phi$.
Thus, if we can find such a potential $\phi$ on the sphere then we know it must define the unique, optimally transported mesh that satisfies the equidistribution principle.
\end{proof}

\begin{corollary}
The optimally transported mesh on the sphere satisfying the equidistribution principle does not exhibit tangling.
\end{corollary}
\begin{proof}
The choice of cost function $c$ to be the squared geodesic distance is crucial to the proof of uniqueness in \autoref{mccannthm}. Indeed simply taking $c$ to be the square of the Euclidean distance is not sufficient \citep{ahmad2004}. The squared geodesic distance is necessary to ensure that the classical \emph{twist} condition holds, i.e. $T$ given in \eqref{t1} is injective and hence is a map.

The injectivity of this map ensures that \eqref{t2} is locally invertible, i.e. for each point in the new mesh $x$ there is a unique point in the original mesh $\xi$ which maps to it - i.e. mesh tangling is not present. 
\end{proof}



\bibliographystyle{apalike}
\bibliography{/home/pbrowne/latex/bib/library,/home/pbrowne/Documents/Mendeley/library,/home/pbrowne/Dropbox/latex/bibliography}
\end{document}
