\begin{frontmatter}

\title{Mesh generation on the sphere using Optimal Transport and the numerical solution of a Monge-Amp\`ere type equation}

%% use optional labels to link authors explicitly to addresses:
\author[label1]{Hilary Weller}
\ead{h.weller@reading.ac.uk}
\author[label1]{Philip Browne}
\author[label2]{Chris Budd}
\author[label3]{Mike Cullen}
\address[label1]{Meteorology, University of Reading, UK}
\address[label2]{University of Bath, UK}
\address[label3]{Met Office, UK}

\begin{abstract}

Graphical abstract\\
\includegraphics[width=\linewidth]{/home/hilary/OpenFOAM/hilary-2.3.0/run/meshes/sphereMeshes/MongeAmpereFromPpt/6/8/pptMesh.png}

An equation of Monge-Amp\`ere type has, for the first time, been solved numerically on the surface of the sphere in order to generate optimally transported (OT) meshes, equidistributed with respect to a monitor function. Optimal transport generates meshes that keep the same connectivity as the original mesh, making them suitable for r-adaptive simulations, in which the equations of motion can be solved in a moving frame of reference in order to avoid mapping the solution between old and new meshes and to avoid load balancing problems on parallel computers. 

The semi-implicit solution of the Monge-Amp\`ere type equation involves a new linearisation of the Hessian term, and exponential maps are used to map from old to new meshes on the sphere. The determinant of the Hessian is evaluated as the change in volume between old and new mesh cells, rather than using numerical approximations to the gradients. 

OT meshes are generated to compare with centroidal Voronoi tesselations on the sphere and are found to have advantages and disadvantages; OT equidistribution is more accurate, the number of iterations to convergence is independent of the mesh size, face skewness is reduced and the connectivity does not change. However anisotropy is higher and the OT meshes are non-orthogonal.

It is shown that optimal transport on the sphere leads to meshes that do not tangle. However, tangling can be introduced by numerical errors in calculating the gradient of the mesh potential. Methods for alleviating this problem are explored. 

Finally, OT meshes are generated using observed precipitation as a monitor function, in order to demonstrate the potential power of the technique.


\end{abstract}

\begin{keyword}
Optimal Transport \sep
Adaptive \sep
Mesh \sep
Refinement \sep
Mesh generation \sep
Monge-Amp\'ere \sep
Atmosphere \sep
Modelling
\end{keyword}

\end{frontmatter}

